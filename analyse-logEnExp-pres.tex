\section{De exponenti\"ele en logaritmische functie}
\frame{\tableofcontents[currentsection]}

\subsection{De exponenti\"ele functie}
\begin{frame}
\frametitle{De exponenti\"ele functie: defintie} 
\pause
\begin{definitie}
\[f:\mathbb{R} \rightarrow \mathbb{R}:x\mapsto a^x\] 
is de {\bf exponenti\"ele functie met grondtal $a$} $(a\in \mathbb{R}_0^+\setminus \{1\})$.
\end{definitie}
~\\
\pause
\begin{itemize}
\item<+-> Indien $a>1$ is de exponenti\"ele functie strikt stijgend. 
\item<+-> Indien $a<1$ is de exponenti\"ele functie strikt dalend. 
\end{itemize}
\end{frame}

\begin{frame}
\frametitle{Eigenschappen}
\pause
\begin{eigenschap}
Stel $f$ de exponenti\"ele functie met grondtal $a$ $(a\in\mathbb{R}_0^{+}\setminus \{1\})$ :
\pause
\begin{itemize}
\item<+-> $f(0)=1$ 
\item<+-> $f(1)=a$ 
\item<+-> $\mbox{dom}\, f=\mathbb{R}$
\item<+-> $\mbox{bld}\, f=\mathbb{R}_0^+$
\item<+-> De functie $f$ is continu in $\mathbb{R}$.
\item<+-> \pause
      Voor $a>1$ geldt:\\
                 de $X$-as is horizontale asymptoot aan de kant van $-\infty$.\\
          \pause
      Voor $0<a<1$ geldt:\\
                de $X$-as is horizontale asymptoot aan de kant van $+\infty$.
\end{itemize}   
\end{eigenschap}
\end{frame}

\begin{frame}  
\frametitle{De natuurlijke exponenti\"ele functie} 
\pause
\begin{definitie}
De {\bfseries natuurlijke exponenti\"ele functie} met {\bfseries natuurlijk grondtal} e is
\[\mbox{exp}:\mathbb{R} \rightarrow \mathbb{R}:x\mapsto \mbox{exp}(x)=\mbox{e}^x\]
met  ($\mbox{e}=2,7182818284\ldots)$. 
\end{definitie}
\end{frame}

\subsection{De logaritmische functie}
\begin{frame}
\frametitle{De logaritmische functie: definitie} 
\pause
%De inverse functie van de exponenti\"ele functie met grondtal $a$ is de {\bf logaritmi\-sche functie met grondtal $a$}.\\~\\
%\pause
%Dus voor\\
%$f:\mathbb{R} \rightarrow \mathbb{R}:x\mapsto y = a^x$\\
%geldt
%\pause
%\[f^{-1}:\mathbb{R} \rightarrow \mathbb{R}:x\mapsto \mbox{log}_a x\]
%met $y=\mbox{log}_a x \Leftrightarrow x=a^y$\\~\\
%\pause
\begin{definitie} 
De {\bfseries logaritmische functie met grondtal $a$} $(a\in \mathbb{R}_0^+\setminus \{1\})$  is
\[f:\mathbb{R} \rightarrow \mathbb{R}:x\mapsto f(x)=y=\mbox{log}_a x\]
met $a^y=x$.
\end{definitie}
~\\
\pause
\begin{itemize}
\item<+-> Indien $0<a<1$ dan is $\mbox{log}_a$ strikt dalend. 
\item<+-> Indien $a>1$ dan is $\mbox{log}_a$ strikt stijgend. 
\end{itemize}
\end{frame}

\begin{frame}
\frametitle{Eigenschappen}
\pause
\begin{eigenschap}
Stel $f$ de logaritmische functie met grondtal $a$ $(a\in\mathbb{R}_0^{+}\setminus \{1\})$:
 
\pause
\begin{itemize}
\item<+-> $f(1)=0$ 
\item<+-> $f(a)=1$ 
\item<+-> $\mbox{dom}\, f=\mathbb{R}^+_0$
\item<+-> $\mbox{bld}\, f=\mathbb{R}$
\item<+-> De functie $f$ is continu in $\mathbb{R}^+_0$
\item<+-> De $Y$-as is verticale asymptoot
\end{itemize}  
\end{eigenschap} 
\end{frame}

\begin{frame}  
\frametitle{Bijzondere logaritmen} 
\pause
\begin{definitie}~
\begin{itemize}
\item<+-> De {\bfseries Briggse logaritmische functie} is de logaritmische functie met grondtal 10
\[\mbox{log}:\mathbb{R}^+_0 \rightarrow \mathbb{R}: x \mapsto y=\mbox{log}\, x.\]
\item<+-> De {\bfseries Neperiaanse of natuur\-lijke logaritmische} functie is de logaritmische functie met grondtal $e$
\[\mbox{ln}:\mathbb{R}^+_0 \rightarrow \mathbb{R}: x \mapsto y=\mbox{ln}\, x.\]
\item<+-> De logaritmische functie met grondtal $2$ wordt genoteerd als $\mbox{lg}\, x$. 
\[\mbox{lg}:\mathbb{R}^+_0 \rightarrow \mathbb{R}: x \mapsto y=\mbox{lg}\, x.\]
\end{itemize}
\end{definitie}
\end{frame}

\begin{frame}
\frametitle{Voorbeelden}
\pause 
\begin{itemize}
\item<+-> $\mbox{log}_2 16 = \ldots $\\~
\item<+-> $\mbox{log}_3 27 = \ldots $\\~
\item<+-> $\mbox{ln}\, e = \ldots$
\end{itemize}
\end{frame}

\begin{frame}
\frametitle{Rekenregels} 
\pause
\begin{eigenschap}
Stel $a,b \in \mathbb{R}_0^+\setminus\{1\}$.\\~\\
\pause 
\[\begin{array}{rcl}
  \mbox{log}_a(x.y) & = & \mbox{log}_a x + \mbox{log}_a y \\~\\\pause
  \mbox{log}_a\displaystyle \frac{x}{y} & = & \mbox{log}_a x - \mbox{log}_a y \\~\\\pause
  \mbox{log}_a(x^p) & = & p.\mbox{log}_a x  ~~~(\mbox{met } p \in \mathbb{Q})\\ ~\\\pause
  \mbox{log}_a a & = & 1 \\~\\ \pause
  \mbox{log}_a x & = & \frac{\mbox{log}_b x}{\mbox{log}_b a}
  \end{array}\]
\end{eigenschap}~\\
\end{frame}

\begin{frame}
\frametitle{Rekenregels: voorbeelden}
\pause
\begin{itemize}
\item<+-> $\mbox{log} 25 + \mbox{log} 4 = \ldots$\\~
\item<+-> $\mbox{ln} 2\mbox{e}^3 - \mbox{ln} 2 = \ldots$\\~
\item<+-> $4.\mbox{log}_9 3 = \ldots$\\~
\item<+-> $\displaystyle \mbox{lg} 12 = \ldots$
\end{itemize} 
\end{frame}

\subsection*{Oefeningen}

\begin{frame}
\frametitle{Oefeningen}
\pause
\begin{enumerate}
\item<+-> Bepaal het domein en beeld van $f$. Geef de nulpunten, het snijpunt met de $Y$-as en eventuele asymptoten van $f$. Teken ten slotte de grafiek van $f$. 
      \begin{enumerate}
      \item<+->[(a)] $f:\mathbb{R}\rightarrow \mathbb{R}:x\mapsto y=\mbox{log}_3\, x$
      \item<+->[(b)] $f:\mathbb{R}\rightarrow \mathbb{R}:x\mapsto y=2+3^x$
      \item<+->[(c)] $f:\mathbb{R}\rightarrow \mathbb{R}:x\mapsto y=-2\mbox{log}_3\, x$
      \end{enumerate}~\\
\item<+-> Stel 
      \[\begin{array}{c}
        f:\mathbb{R}\rightarrow \mathbb{R}: x\mapsto \mbox{ln(exp}(x))\\
        g:\mathbb{R}\rightarrow \mathbb{R}: x\mapsto \mbox{exp(ln}(x)).
        \end{array}\]
      Wat is het verschil tussen de functie $f$ en de functie $g$?\\
\end{enumerate}
\end{frame}

\begin{frame}
%\frametitle{Oefeningen}
\pause
\small{
\begin{enumerate}
\item<+->[3]
Je speelt het spel Angry Birds. Je moet door boze vogels te lanceren zoveel mogelijk groene varkens raken. Je krijgt hiervoor drie pogingen. Er bevinden zich groene varkens op de punten: $(1,4)$, $(2,4)$, $(2,5)$, $(3,1)$, $(3,3)$, $(3,4)$, $(3,5)$, $(4,2)$, $(4,4)$. \\
Je mag drie vogels lanceren: een bruine, rode en blauwe vogel. 
\begin{itemize}
\item De bruine vogel start in het punt met co\"ordinaten $(1,3)$ en vliegt volgens de grafiek van de functie $f$ met $f(x)= 2^x+1$.
\item De rode vogel start in het punt met co\"ordinaten $(2,0)$ en vliegt volgens de grafiek van de functie $g$ met $g(x)=\lg (x-1)$.
\item De blauwe vogel start in het punt met co\"ordinaten $(2,3)$ en vliegt volgens de grafiek van de functie $h$ met $h(x)=-x^2+6x-5$.
\end{itemize}

Beantwoord de volgende vragen:
\begin{enumerate}
\item[a)] Maak een tekening van de gegeven situatie. Duid eveneens de vlucht van elke vogel op jouw tekening aan. 
%\item[b)] Raakt elke vogel een varken? Zo ja, welk varken wordt er geraakt door welke vogel? \\
%Motiveer jouw antwoord a.d.h.v.\ een analytische berekening. Verifieer vervolgens of jouw berekeningen %overeenstemmen met jouw tekening.
%\item[c)] De varkens met co\"ordinaten $(2,4)$ en $(4,2)$ zijn nog niet geraakt. Je krijgt twee extra vogels om specifiek deze varkens uit te schakelen. Beide vogels vertrekken van het punt met co\"ordinaten $(1,2)$. Ze vliegen allebei volgens een rechte lijn. \\
%Geef voor beide vogels de vergelijking van de functie die hun vlucht volgt.
\end{enumerate}
\end{enumerate}}
\end{frame}

\begin{frame}
%\frametitle{Oefeningen}
%\pause
\small{
\begin{enumerate}
\item<+->[3]
Je speelt het spel Angry Birds. Je moet door boze vogels te lanceren zoveel mogelijk groene varkens raken. Je krijgt hiervoor drie pogingen. Er bevinden zich groene varkens op de punten: $(1,4)$, $(2,4)$, $(2,5)$, $(3,1)$, $(3,3)$, $(3,4)$, $(3,5)$, $(4,2)$, $(4,4)$. \\
Je mag drie vogels lanceren: een bruine, rode en blauwe vogel. 
\begin{itemize}
\item De bruine vogel start in het punt met co\"ordinaten $(1,3)$ en vliegt volgens de grafiek van de functie $f$ met $f(x)= 2^x+1$.
\item De rode vogel start in het punt met co\"ordinaten $(2,0)$ en vliegt volgens de grafiek van de functie $g$ met $g(x)=\lg (x-1)$.
\item De blauwe vogel start in het punt met co\"ordinaten $(2,3)$ en vliegt volgens de grafiek van de functie $h$ met $h(x)=-x^2+6x-5$.
\end{itemize}

Beantwoord de volgende vragen:
\begin{enumerate}
%\item[a)] Maak een tekening van de gegeven situatie. Duid eveneens de vlucht van elke vogel op jouw tekening aan. 
\item[b)] Raakt elke vogel een varken? Zo ja, welk varken wordt er geraakt door welke vogel? \\
Motiveer jouw antwoord a.d.h.v.\ een analytische berekening. Verifieer vervolgens of jouw berekeningen overeenstemmen met jouw tekening.
%\item[c)] De varkens met co\"ordinaten $(2,4)$ en $(4,2)$ zijn nog niet geraakt. Je krijgt twee extra vogels om specifiek deze varkens uit te schakelen. Beide vogels vertrekken van het punt met co\"ordinaten $(1,2)$. Ze vliegen allebei volgens een rechte lijn. \\
%Geef voor beide vogels de vergelijking van de functie die hun vlucht volgt.
\end{enumerate}
\end{enumerate}}
\end{frame}

\begin{frame}
%\frametitle{Oefeningen}
%\pause
\small{
\begin{enumerate}
\item<+->[3]
Je speelt het spel Angry Birds. Je moet door boze vogels te lanceren zoveel mogelijk groene varkens raken. Je krijgt hiervoor drie pogingen. Er bevinden zich groene varkens op de punten: $(1,4)$, $(2,4)$, $(2,5)$, $(3,1)$, $(3,3)$, $(3,4)$, $(3,5)$, $(4,2)$, $(4,4)$. \\
Je mag drie vogels lanceren: een bruine, rode en blauwe vogel. 
\begin{itemize}
\item De bruine vogel start in het punt met co\"ordinaten $(1,3)$ en vliegt volgens de grafiek van de functie $f$ met $f(x)= 2^x+1$.
\item De rode vogel start in het punt met co\"ordinaten $(2,0)$ en vliegt volgens de grafiek van de functie $g$ met $g(x)=\lg (x-1)$.
\item De blauwe vogel start in het punt met co\"ordinaten $(2,3)$ en vliegt volgens de grafiek van de functie $h$ met $h(x)=-x^2+6x-5$.
\end{itemize}

Beantwoord de volgende vragen:
\begin{enumerate}
%\item[a)] Maak een tekening van de gegeven situatie. Duid eveneens de vlucht van elke vogel op jouw tekening aan. 
%\item[b)] Raakt elke vogel een varken? Zo ja, welk varken wordt er geraakt door welke vogel? \\
%Motiveer jouw antwoord a.d.h.v.\ een analytische berekening. Verifieer vervolgens of jouw berekeningen overeenstemmen met jouw tekening.
\item[c)] De varkens met co\"ordinaten $(2,4)$ en $(4,2)$ zijn nog niet geraakt. Je krijgt twee extra vogels om specifiek deze varkens uit te schakelen. Beide vogels vertrekken van het punt met co\"ordinaten $(1,2)$. Ze vliegen allebei volgens een rechte lijn. \\
Geef voor beide vogels de vergelijking van de functie die hun vlucht volgt.
\end{enumerate}
\end{enumerate}}
\end{frame}